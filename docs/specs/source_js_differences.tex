\section*{Deviations from JavaScript}

We intend the 
Source language to be a conservative extension
of JavaScript: Every correct
Source program should behave \emph{exactly} the same
using a Source implementation, as it does using a JavaScript
implementation. We assume, of course, that suitable libraries are
used by the JavaScript implementation, to account for the predefined names
of each Source language.

This section lists some exceptions where we think a Source implementation
should be allowed to deviate from the JavaScript specification, for the
sake of internal consistency and esthetics.

\begin{description}
\item[\href{https://source-academy.github.io/sicp/chapters/4.1.1.html\#footnote-4}{Evaluation result of programs:}] In JavaScript,
  the specification of the result of evaluating a program is surprisingly
  subtle. ECMAScript distinguishes between value-producing statements
  (such as expression statements, assignments and conditional statements)
  and non-value-producing statements such as declarations. A block is
  value-producing if any of the statements that its body is made of
  are value-producing. The value of a conditional statement is the
  value of its executed branch, regardless whether that branch is
  value-producing or not; if that branch is not value-producing, the value of
  the conditional statement is the value \texttt{undefined}.
  A program  has as value the value of the last value-producing statement
  that it is composed of, and \texttt{undefined} if there is no
  such statement.

  Example 1:
  \begin{lstlisting}
1;
{
  // empty block
}
  \end{lstlisting}
  The result of evaluating this program in JavaScript is 1.

  Example 2:
  \begin{lstlisting}
1;
{
  if (true) {} else {}
}
  \end{lstlisting}
  The result of evaluating this program in JavaScript is \texttt{undefined}.

  Implementations of Source may opt for a simpler scheme. Most implementations
  of Source in the Source Academy comply with this regime, with the current
  exception of the Stepper.

\item[\href{https://source-academy.github.io/sicp/chapters/1.3.2.html\#footnote-2}{Hoisting of function declarations:}] In JavaScript, function declarations
  are ``hoisted''
  (\href{https://source-academy.github.io/sicp/chapters/4.3.1.html#footnote-4}{automagically} moved)
  to the beginning of the block in which
  they appear. This means that applications of functions that are declared
  with function declaration statements never fail because the name is not
  yet assigned to their function value. The specification of Source does 
  not include this hoisting; in Source, function declaration can be seen as
  syntactic sugar for constant declaration and lambda expression.
  As a consequence, application of functions declared with function declaration
  may fail in Source if the name that appears as function expression is not yet
  assigned to the function value it is supposed to refer to.
\end{description}
